
\documentclass[]{article}
	
\usepackage[margin=1in]{geometry}		% For setting margins
\usepackage{amsmath}
\usepackage{amssymb}
\usepackage{fancyhdr}
\usepackage{graphicx}
\usepackage{cancel}
\usepackage{tikz}
\usetikzlibrary{quantikz2}

%%%%%%%%%%%%%%%%%%%%%%
% Set up fancy header/footer
\pagestyle{fancy}
\fancyhead[LO,L]{Anthony J. Lombardi}
\fancyhead[CO,C]{CS569: Quantum Info and Computing}
\fancyhead[RO,R]{\today}
\fancyfoot[LO,L]{}
\fancyfoot[CO,C]{\thepage}
\fancyfoot[RO,R]{}
\renewcommand{\headrulewidth}{0.4pt}
\renewcommand{\footrulewidth}{0.4pt}
%%%%%%%%%%%%%%%%%%%%%%

\begin{document}
  \section*{Assignment 2}
    \begin{enumerate}
      % Question 1
      \item Consider the Bell state $\ket{\beta_{00}}$, suppose both qubits are measured in the diagonal basis. 
        What are the outcomes?

        We know that $\ket{\beta_{00}}$ is equivalent to $\frac{1}{\sqrt{2}}(\ket{00} + \ket{11})$ in order 
        to mesaure in the diagonal basis. To do this, we need to express $\ket{\beta_{00}}$ 
        in terms of $\{\ket{+}, \ket{-}\}$ rather than $\{\ket{0}, \ket{1}\}$.
        Luckily we know that $\ket{0} = \frac{1}{\sqrt{2}}(\ket{+} + \ket{-})$ and that
        $\ket{1} = \frac{1}{\sqrt{2}}(\ket{+} - \ket{-})$\footnote{T. Wong, Introduction to 
        Classical and Quantum Computing, Rooted Grove, 2022. Page 87}. 

        Subsituting this questions into $\ket{\beta_{00}}$:

        \begin{gather*}
          \frac{1}{\sqrt{2}}(\ket{00} + \ket{11}) \\
          = \frac{1}{\sqrt{2}}[(\frac{1}{\sqrt{2}}(\ket{+} + \ket{-}) \otimes \frac{1}{\sqrt{2}}(\ket{+} + \ket{-}))
          + (\frac{1}{\sqrt{2}}(\ket{+} - \ket{-}) \otimes \frac{1}{\sqrt{2}}(\ket{+} - \ket{-}))] \\
          \\
          = \frac{1}{\sqrt{2}}[\frac{1}{2}((\ket{+} + \ket{-}) \otimes (\ket{+} + \ket{-}))
          + \frac{1}{2}((\ket{+} - \ket{-}) \otimes (\ket{+} - \ket{-}))] \\
          \\
          = \frac{1}{\sqrt{2}}\frac{1}{2}[(\ket{++} + \ket{+-} + \ket{-+} + \ket{--})
          + (\ket{++} - \ket{+-} - \ket{-+} + \ket{--})] \\
          \\
          = \frac{1}{\sqrt{2}}[(\ket{++} + \ket{--})] \\
        \end{gather*}

        % Now we need to mesaure each qubit
      
      % Question 2
      \item Describe a quantum teleportation circuit when ALice and Bob shared the Bell state
        $\ket{\beta_{11}}$ instead of $\ket{\beta_{00}}$.

        The following circuit will successfully teleport an arbitrary qubit, $\ket{\psi}$, between Alice and Bob.

        \begin{quantikz}
          \lstick{\ket{\psi}} & \ctrl{1} & \gate{H} & \meter{} & \setwiretype{c}&  \gate{NOT} && \ctrl[vertical wire=c]{2}\\ 
          \lstick[2]{\ket{\beta_{11}}} & \targ{} && \meter{} & \setwiretype{c} & \gate{NOT} & \ctrl[vertical wire=c]{1}\\ 
                                       & &&& && \gate{x}& \gate{z} & \rstick{\ket{\psi}}
        \end{quantikz}

        Proof:

        \begin{gather*}
          \ket{\psi} \otimes \ket{\beta_{11}} = (a\ket{0} + b\ket{1}) \otimes \frac{\ket{01} - \ket{10}}{\sqrt{2}} \\
          = a\ket{0} \otimes \frac{\ket{01} - \ket{10}}{\sqrt{2}} + 
          b\ket{1} \otimes \frac{\ket{01} - \ket{10}}{\sqrt{2}} \\
          = \frac{1}{\sqrt{2}}[a\ket{001} - a\ket{010} + b\ket{101} - b\ket{110}]\\
          \xrightarrow{\text{CNOT}} \frac{1}{\sqrt{2}}[a\ket{001} - a\ket{010} + b\ket{111} - b\ket{100}]\\
          \xrightarrow{H \otimes I \otimes I}
          \frac{1}{\sqrt{2}}[a \frac{\ket{0} + \ket{1}}{\sqrt{2}}(\ket{01} - \ket{10}) + 
          b\frac{\ket{0} - \ket{1}}{\sqrt{2}} (\ket{11} - \ket{00})]\\
          = \frac{1}{2}[(a\ket{0} + a\ket{1})(\ket{01} - \ket{10}) + 
          (b\ket{0} - b\ket{1})(\ket{11} - \ket{00})]\\
          = \frac{1}{2}[a\ket{001} - a\ket{010} + a\ket{101} - a\ket{110} +
          b\ket{011} - b\ket{000} + b\ket{111} - b\ket{100}] \\
          = \frac{1}{2}[\ket{00}(a\ket{1}-b\ket{0}) - \ket{01}(a\ket{0} - b\ket{1})
          + \ket{10}(a\ket{1} + b\ket{0}) - \ket{11}(a\ket{0} + b\ket{1})]\\
        \end{gather*}

        Now messuring the first two qubits gets us:

        \[ \begin{cases} 
          \ket{00} XZ\ket{\psi} & \text{w.p. } \frac{1}{4} \\
          \ket{01} Z\ket{\psi} & \text{w.p. } \frac{1}{4} \\
          \ket{10} X\ket{\psi} & \text{w.p. } \frac{1}{4} \\
          \ket{11} \ket{\psi} & \text{w.p. } \frac{1}{4} \\
          \end{cases}
        \]


    \end{enumerate}

\end{document} 
