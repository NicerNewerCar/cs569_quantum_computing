\documentclass[]{article}
	
\usepackage[margin=1in]{geometry}		% For setting margins
\usepackage{amsmath}
\usepackage{amssymb}
\usepackage{fancyhdr}
\usepackage{graphicx}
\usepackage{cancel}
\usepackage{tikz}
\usetikzlibrary{quantikz2}

%%%%%%%%%%%%%%%%%%%%%%
% Set up fancy header/footer
\pagestyle{fancy}
\fancyhead[LO,L]{Anthony J. Lombardi}
\fancyhead[CO,C]{CS569: Quantum Info and Computing}
\fancyhead[RO,R]{\today}
\fancyfoot[LO,L]{}
\fancyfoot[CO,C]{\thepage}
\fancyfoot[RO,R]{}
\renewcommand{\headrulewidth}{0.4pt}
\renewcommand{\footrulewidth}{0.4pt}
%%%%%%%%%%%%%%%%%%%%%%

\begin{document}
  \section*{Assignment 3}
    \begin{enumerate}
      % Question 1
      \item Finding a hiden parity function:
        \begin{enumerate}
          \item Truth table of $f$ and its $\pm$ 1 version: \\ 
            \begin{table}[h]
              \centering
            \begin{tabular}{ll|l|l}
            $x_1$ & $x_2$ & f & F  \\ \hline
            0   & 0   & 1 & -1 \\
            0   & 1   & 0 & 1  \\
            1   & 0   & 1 & -1 \\
            1   & 1   & 0 & 1 
            \end{tabular}
            \end{table}

          \item Truth table of the Walsh basis function: \\ 
            \begin{table}[h]
            \centering
            \begin{tabular}{l|llll}
                     & \ket{00} & \ket{01} & \ket{10} & \ket{11} \\ \hline
            \ket{00} & 1        & 1        & 1        & 1        \\
            \ket{01} & 1        & -1       & 1        & -1       \\
            \ket{10} & 1        & 1        & -1       & -1       \\
            \ket{11} & 1        & -1       & -1       & 1       
            \end{tabular}
            \end{table}

          \item F as a linear combination of the Walsh basis functions:
          \item Complete trace
          \item The output ...
        \end{enumerate} 


    \end{enumerate}

\end{document} 
