\documentclass[]{article}
	
\usepackage[margin=1in]{geometry}		% For setting margins
\usepackage{amsmath}
\usepackage{amssymb}
\usepackage{fancyhdr}
\usepackage{graphicx}
\usepackage{cancel}
\usepackage{tikz}
\usetikzlibrary{quantikz2}

%%%%%%%%%%%%%%%%%%%%%%
% Set up fancy header/footer
\pagestyle{fancy}
\fancyhead[LO,L]{Anthony J. Lombardi}
\fancyhead[CO,C]{CS569: Quantum Info and Computing}
\fancyhead[RO,R]{\today}
\fancyfoot[LO,L]{}
\fancyfoot[CO,C]{\thepage}
\fancyfoot[RO,R]{}
\renewcommand{\headrulewidth}{0.4pt}
\renewcommand{\footrulewidth}{0.4pt}
%%%%%%%%%%%%%%%%%%%%%%

\begin{document}
  \section*{Assignment 3}
    \begin{enumerate}
      % Question 1
      \item Finding a hidden parity function:
        \begin{enumerate}
          \item Truth table of $f$ and its $\pm$ 1 version: \\ 
            \begin{table}[h]
              \centering
            \begin{tabular}{ll|l|l}
            $x_1$ & $x_2$ & f & F  \\ \hline
            0   & 0   & 1 & -1 \\
            0   & 1   & 0 & 1  \\
            1   & 0   & 1 & -1 \\
            1   & 1   & 0 & 1 
            \end{tabular}
            \end{table}

          \item Truth table of the Walsh basis function: \\ 
            \begin{table}[h]
            \centering
            \begin{tabular}{l|llll}
                     & \ket{00} & \ket{01} & \ket{10} & \ket{11} \\ \hline
            \ket{00} & 1        & 1        & 1        & 1        \\
            \ket{01} & 1        & -1       & 1        & -1       \\
            \ket{10} & 1        & 1        & -1       & -1       \\
            \ket{11} & 1        & -1       & -1       & 1       
            \end{tabular}
            \end{table}

          \item F as a linear combination of the Walsh basis functions:
          \item Complete trace
            \begin{gather*}
              \ket{00} \xrightarrow{H \otimes H} \frac{1}{2}\Sigma_{x \in \{0,1\}^2} \ket{x}
              \xrightarrow{\hat{U}_f} \frac{1}{2}\Sigma_{x \in \{0,1\}^2} (-1)^{f(x)} \ket{x}\\
              \xrightarrow{H \otimes H} \frac{1}{4} \Sigma_{y \in \{0,1\}^2} (
              \Sigma_{x \in \{0,1\}^2} (-1)^{f{x}} (-1)^{y \cdot x}) \ket{y}\\
              \text{Let } F(x) = (-1)^{f(x)}\\
              \text{Let } \chi_y(x) = (-1)^{y \cdot x}\\
              =  \Sigma_{y \in \{0,1\}^2} (\frac{1}{4}
              \Sigma_{x \in \{0,1\}^2} F(x)\chi_y(x) ) \ket{y}\\
            \end{gather*}

            We can now use the tables from parts (a) and (b) to solve the equation:
            \begin{gather*}
              = [\frac{1}{4}[
                F(\ket{00})\chi_{\ket{00}(\ket{00})}+
                F(\ket{01})\chi_{\ket{00}(\ket{01})}+
                F(\ket{10})\chi_{\ket{00}(\ket{10})}+
                F(\ket{11})\chi_{\ket{00}(\ket{11})}
              ]\ket{00}+ \\
              \frac{1}{4}[
                F(\ket{00})\chi_{\ket{01}(\ket{00})}+
                F(\ket{01})\chi_{\ket{01}(\ket{01})}+
                F(\ket{10})\chi_{\ket{01}(\ket{10})}+
                F(\ket{11})\chi_{\ket{01}(\ket{11})}
              ]\ket{01}+\\
              \frac{1}{4}[
                F(\ket{00})\chi_{\ket{10}(\ket{00})}+
                F(\ket{01})\chi_{\ket{10}(\ket{01})}+
                F(\ket{10})\chi_{\ket{10}(\ket{10})}+
                F(\ket{11})\chi_{\ket{10}(\ket{11})}             
              ]\ket{10}+\\
              \frac{1}{4}[
                F(\ket{00})\chi_{\ket{11}(\ket{00})}+
                F(\ket{01})\chi_{\ket{11}(\ket{01})}+
                F(\ket{10})\chi_{\ket{11}(\ket{10})}+
                F(\ket{11})\chi_{\ket{11}(\ket{11})}
              ]\ket{11}] \\
              = [
              \frac{1}{4}[-1 \cdot 1 + 1 \cdot 1 + -1 \cdot 1 + 1 \cdot 1]\ket{00}+
              \frac{1}{4}[-1 \cdot 1 + 1 \cdot -1 + -1 \cdot 1 + 1 \cdot -1]\ket{01}+\\
              \frac{1}{4}[-1 \cdot 1 + 1 \cdot 1 + -1 \cdot -1 + 1 \cdot -1]\ket{10}+
              \frac{1}{4}[-1 \cdot 1 + 1 \cdot -1 + -1 \cdot -1 + 1 \cdot 1]\ket{11}] \\
              = [0 \cdot \ket{00} - 1 \cdot \ket{01} + 0 \cdot \ket{10} + 0 \cdot \ket{11}]
              = \ket{01}
            \end{gather*}
          \item (T/F) The output of the circuit reveals what f is.

            False the output does not reveal what f is.
        \end{enumerate} 

      \pagebreak
      % Question 2
      \item Finding a hidden shift:
        \begin{enumerate}
          \item Show the truth table f and its $\pm$ version:
            \begin{table}[h]
              \centering
            \begin{tabular}{ll|l|l}
            $x_1$ & $x_2$ & f & F  \\ \hline
            0   & 0   & 0 & 1 \\
            0   & 1   & 1 & -1 \\
            1   & 0   & 1 & -1 \\
            1   & 1   & 0 & 1
            \end{tabular}
            \end{table}

          \item Complete trace:
            \begin{gather*}
              \ket{00} \xrightarrow{H \otimes H} \frac{1}{2}\Sigma_{x \in \{0,1\}^2} \ket{x}
              \xrightarrow{\hat{U}_f} \frac{1}{2}\Sigma_{x \in \{0,1\}^2} (-1)^{f(x)} \ket{x}\\
              \xrightarrow{H \otimes H} \frac{1}{4} \Sigma_{y \in \{0,1\}^2} (
              \Sigma_{x \in \{0,1\}^2} (-1)^{f{x}} (-1)^{y \cdot x}) \ket{y}\\
              \text{Let } F(x) = (-1)^{f(x)}\\
              \text{Let } \chi_y(x) = (-1)^{y \cdot x}\\
              =  \Sigma_{y \in \{0,1\}^2} (\frac{1}{4}
              \Sigma_{x \in \{0,1\}^2} F(x)\chi_y(x) ) \ket{y}\\
            \end{gather*}

            We can now use the table from part (a), and our knowledge of $\chi$, to solve the equation:
            \begin{gather*}
              = [\frac{1}{4}[
                F(\ket{00})\chi_{\ket{00}(\ket{00})}+
                F(\ket{01})\chi_{\ket{00}(\ket{01})}+
                F(\ket{10})\chi_{\ket{00}(\ket{10})}+
                F(\ket{11})\chi_{\ket{00}(\ket{11})}
              ]\ket{00}+ \\
              \frac{1}{4}[
                F(\ket{00})\chi_{\ket{01}(\ket{00})}+
                F(\ket{01})\chi_{\ket{01}(\ket{01})}+
                F(\ket{10})\chi_{\ket{01}(\ket{10})}+
                F(\ket{11})\chi_{\ket{01}(\ket{11})}
              ]\ket{01}+\\
              \frac{1}{4}[
                F(\ket{00})\chi_{\ket{10}(\ket{00})}+
                F(\ket{01})\chi_{\ket{10}(\ket{01})}+
                F(\ket{10})\chi_{\ket{10}(\ket{10})}+
                F(\ket{11})\chi_{\ket{10}(\ket{11})}             
              ]\ket{10}+\\
              \frac{1}{4}[
                F(\ket{00})\chi_{\ket{11}(\ket{00})}+
                F(\ket{01})\chi_{\ket{11}(\ket{01})}+
                F(\ket{10})\chi_{\ket{11}(\ket{10})}+
                F(\ket{11})\chi_{\ket{11}(\ket{11})}
              ]\ket{11}] \\
              = [
              \frac{1}{4}[1 \cdot 1 + -1 \cdot 1 + -1 \cdot 1 + 1 \cdot 1] \ket{00} + 
              \frac{1}{4}[1 \cdot 1 + -1 \cdot -1 + -1 \cdot 1 + 1 \cdot -1] \ket{01} +\\ 
              \frac{1}{4}[1 \cdot 1 + -1 \cdot 1 + -1 \cdot -1 + 1 \cdot -1] \ket{10} + 
              \frac{1}{4}[1 \cdot 1 + -1 \cdot -1 + -1 \cdot -1 + 1 \cdot 1] \ket{11}]\\
              = [0 \cdot \ket{00} + 0 \cdot \ket{01} + 0 \cdot \ket{10} - 1\cdot \ket{11}]
              = \ket{11}
            \end{gather*}

          \item The final output of the circuit is $\ket{00}$.
          \item (T/F) The circuit reveals the hidden string s?

            True the circuit does output the hidden string.
        \end{enumerate}

      % Question 3
      \item Quantum Fourier Transform 
        \begin{enumerate}
          \item (T/F) The following bit-level QFT derivation for $N = 4$ is correct.

            True the derivation for $N = 4$ is correct.

          \item (T/F) $R_k H \ket{0} = \frac{(\ket{0} + e^{\frac{2\pi i}{2^k}} \ket{1})}{\sqrt{2}}$

            \begin{gather*}
              R_k H \ket{0} =
              \begin{pmatrix}
                1 & 0\\
                0 & e^{\frac{2\pi i }{2^k}}
              \end{pmatrix}
              \begin{pmatrix}
                \frac{1}{\sqrt{2}} & \frac{1}{\sqrt{2}} \\
                \frac{1}{\sqrt{2}} & -\frac{1}{\sqrt{2}} 
              \end{pmatrix}
              \begin{pmatrix}
                1\\
                0
              \end{pmatrix}\\
              = \begin{pmatrix}
                1 \cdot \frac{1}{\sqrt{2}} + 0 \cdot \frac{1}{\sqrt{2}} & 
                1 \cdot \frac{1}{\sqrt{2}} + 0 \cdot - \frac{1}{\sqrt{2}}\\
                0 \cdot \frac{1}{\sqrt{2}} + e^{\frac{2 \pi i}{2^k}} \cdot \frac{1}{\sqrt{2}} &
                0 \cdot \frac{1}{\sqrt{2}} + e^{\frac{2 \pi i}{2^k}} \cdot - \frac{1}{\sqrt{2}}
              \end{pmatrix} 
              \begin{pmatrix}
                1\\
                0
              \end{pmatrix}\\
              = \begin{pmatrix}
                \frac{1}{\sqrt{2}} & \frac{1}{\sqrt{2}}\\
                \frac{e^{\frac{2 \pi i}{2^k}}}{\sqrt{2}} &  - \frac{e^{\frac{2 \pi i}{2^k}}}{\sqrt{2}}
              \end{pmatrix}
              \begin{pmatrix}
                1\\
                0
              \end{pmatrix} = 
              \begin{pmatrix}
                \frac{1}{\sqrt{2}} \cdot 1 + \frac{1}{\sqrt{2}} \cdot 0\\
                \frac{e^{\frac{2 \pi i}{2^k}}}{\sqrt{2}} \cdot 1 + - \frac{e^{\frac{2 \pi i}{2^k}}}{\sqrt{2}} \cdot 0
              \end{pmatrix} = 
              \begin{pmatrix}
                \frac{1}{\sqrt{2}}\\ 
                \frac{e^{\frac{2 \pi i}{2^k}}}{\sqrt{2}} 
              \end{pmatrix} \\= 
              \frac{1}{\sqrt{2}}
              \begin{pmatrix}
                1 \\
                e^{\frac{2 \pi i}{2^k}}
              \end{pmatrix}
              = \frac{(\ket{0} + e^{\frac{2\pi i}{2^k}} \ket{1})}{\sqrt{2}}
            \end{gather*}
            True. //
          \item (T/F) The following is a correct QFT circuit for N = 4.

            True, the circuit is correct because $N = 4 = 2^2$ so $n=2$. Where (starting from the bottom wire) the 
            pattern of gates is as follows $H \Sigma_{i = 2}^n R_i$ for all lines $\ket{j_n} \dots \ket{j_1}$. Then
            for line $\ket{j_0}$ just the Hadamard gate is applied.
        \end{enumerate}
    \end{enumerate}

\end{document} 

