\documentclass[]{article}
	
\usepackage[margin=1in]{geometry}		% For setting margins
\usepackage{amsmath}				% For Math
\usepackage{fancyhdr}				% For fancy header/footer
\usepackage{graphicx}				% For including figure/image
\usepackage{cancel}					% To use the slash to cancel out stuff in work
\usepackage{tikz}
\usetikzlibrary{quantikz2}

%%%%%%%%%%%%%%%%%%%%%%
% Set up fancy header/footer
\pagestyle{fancy}
\fancyhead[LO,L]{Anthony J. Lombardi}
\fancyhead[CO,C]{CS569: Quantum Info and Computing}
\fancyhead[RO,R]{\today}
\fancyfoot[LO,L]{}
\fancyfoot[CO,C]{\thepage}
\fancyfoot[RO,R]{}
\renewcommand{\headrulewidth}{0.4pt}
\renewcommand{\footrulewidth}{0.4pt}
%%%%%%%%%%%%%%%%%%%%%%

\begin{document}
\section*{Assignment 1}
\begin{enumerate}
  % Question 4
  \item Describe a quantum circuit that starting with \ket{000} prepares the output state:

  \begin{equation*}
    \ket{\psi} = \frac{1}{\sqrt{2}}(\ket{000} + \ket{111}) 
  \end{equation*}

  The following 3 stage circuit will produce the desiried output state:
  \begin{center}
    \begin{quantikz}
      & \gate{H} \slice{1} & \ctrl{1} \slice{2} & & & \\
      & & \targ{} & \ctrl{1} \slice{3} & & \\
      & & & \targ{} & &
    \end{quantikz}
  \end{center}

  \begin{itemize}
    \item Stage 1 - Hadamard Gate: 
      \begin{align*}
        \ket{0} \otimes \ket{0} \otimes \ket{0} \xrightarrow{H \otimes I \otimes I} \frac{\ket{0} \otimes \ket{1}}{\sqrt{2}} \otimes \ket{0} \otimes \ket{0} \\
        = \frac{1}{\sqrt{2}}(\ket{000} + \ket{100}) 
      \end{align*}
    \item Stage 2 - 1$^{\text{st}}$ Control Not:

      Just looking at the first two qubits of the superposition. The first qubit is the control bit and second qubit is the bit we may modifiy.
      \begin{align*}
        & \frac{1}{\sqrt{2}}(\ket{000} + \ket{100}) \xrightarrow{\text{CNOT}} \frac{1}{\sqrt{2}}(\ket{000} + \ket{110}) 
      \end{align*}
    \item Stage 3 - 2$^{\text{nd}}$ Control Not:
      
      Just looking at the last two qubits of the superposition.  The second qubit is the control bit and the third qubit is the bit we may modify
      \begin{align*}
        & \frac{1}{\sqrt{2}}(\ket{000} + \ket{110}) \xrightarrow{\text{CNOT}} \frac{1}{\sqrt{2}}(\ket{000} + \ket{111}) 
      \end{align*}
  \end{itemize}


\end{enumerate}
\end{document}
