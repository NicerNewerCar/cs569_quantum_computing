\documentclass[]{article}
	
\usepackage[margin=1in]{geometry}		% For setting margins
\usepackage{amsmath}
\usepackage{amssymb}
\usepackage{fancyhdr}
\usepackage{graphicx}
\usepackage{cancel}
\usepackage{tikz}
\usetikzlibrary{quantikz2}

%%%%%%%%%%%%%%%%%%%%%%
% Set up fancy header/footer
\pagestyle{fancy}
\fancyhead[LO,L]{Anthony J. Lombardi}
\fancyhead[CO,C]{CS569: Quantum Info and Computing}
\fancyhead[RO,R]{\today}
\fancyfoot[LO,L]{}
\fancyfoot[CO,C]{\thepage}
\fancyfoot[RO,R]{}
\renewcommand{\headrulewidth}{0.4pt}
\renewcommand{\footrulewidth}{0.4pt}
%%%%%%%%%%%%%%%%%%%%%%

\begin{document}
\section*{Assignment 1}
\begin{enumerate}
  % Question 1
 \item Show that the Hadamard gate \textit{almost} commutes with the Pauli gates.

   Using the Pauli gates X and Z we can show that the Hadamard gate almost commutes.
   \begin{center}
    X*H = H*Z \\

    \[
    X*H = 
      \begin{bmatrix}
        0 & 1 \\
        1 & 0 \\
      \end{bmatrix}
     * 
     \begin{bmatrix}
       1 & 1 \\
       1 & -1 \\
     \end{bmatrix}
     = 
     \begin{bmatrix}
       1 & -1 \\
       1 & 1 \\
     \end{bmatrix}
    \]

    \[
    H*Z= 
     \begin{bmatrix}
       1 & 1 \\
       1 & -1 \\
     \end{bmatrix}
     *
     \begin{bmatrix}
       1 & 0 \\
       0 & -1 \\
     \end{bmatrix}
     = 
     \begin{bmatrix}
       1 & -1 \\
       1 & 1 \\
     \end{bmatrix}
    \]
   \end{center}


  % Question 2
  \item Show that the Bell states form a basis of all two-qubit states.

    \begin{quantikz}
      & \gate{H} & \ctrl{1} & \\
      & & \targ{} &
    \end{quantikz}

        Bell states $\ket{\beta_{00}}$ and $\ket{\beta_{10}}$ can be combind to form $\ket{00}$.

        \begin{align*}
          \ket{00} = \ket{\beta_{00}} + \ket{\beta_{10}} \\
          = \frac{\ket{00} + \ket{11}}{\sqrt{2}} + \frac{\ket{00} - \ket{11}}{\sqrt{2}} \\
          = \frac{1}{\sqrt{2}} \ket{00} + \frac{1}{\sqrt{2}} \ket{11} + \frac{1}{\sqrt{2}} \ket{00} - \frac{1}{\sqrt{2}} \ket{11} \\ 
          = \frac{1}{\sqrt{2}} \ket{00} + \frac{1}{\sqrt{2}} \ket{00} + \frac{1}{\sqrt{2}} \ket{11} - \frac{1}{\sqrt{2}} \ket{11} \\  
          = \frac{2}{\sqrt{2}} \ket{00} + 0 \ket{11} \\
        \end{align*}

        Normalize the qubit.
        \begin{align*}
          = 1 \ket{00} + 0 \ket{11} \\ 
          = \ket{00}
        \end{align*}

        Bell states $\ket{\beta_{00}}$ and $\ket{\beta_{10}}$ can be combind to form $\ket{11}$.

        \begin{align*}
          \ket{11} = \ket{\beta_{00}} - \ket{\beta_{10}} \\
          = \frac{\ket{00} + \ket{11}}{\sqrt{2}} - \frac{\ket{00} - \ket{11}}{\sqrt{2}} \\
          = \frac{1}{\sqrt{2}} \ket{00} + \frac{1}{\sqrt{2}} \ket{11} - (\frac{1}{\sqrt{2}} \ket{00} - \frac{1}{\sqrt{2}} \ket{11}) \\ 
          = \frac{1}{\sqrt{2}} \ket{00} + \frac{1}{\sqrt{2}} \ket{11} - \frac{1}{\sqrt{2}} \ket{00} + \frac{1}{\sqrt{2}} \ket{11} \\ 
          = \frac{1}{\sqrt{2}} \ket{00} - \frac{1}{\sqrt{2}} \ket{00} + \frac{1}{\sqrt{2}} \ket{11} + \frac{1}{\sqrt{2}} \ket{11} \\  
          = 0 \ket{00} + \frac{2}{\sqrt{2}} \ket{11}\\ 
        \end{align*}

        Normalize the qubit.
        \begin{align*}
          = 0 \ket{00} + 1 \ket{11} \\ 
          = \ket{11}
        \end{align*}

        Similary, $\ket{\beta_{01}}$ and $\ket{\beta_{11}}$ can be combind to form $\ket{01}$ and $\ket{10}$.

        \begin{align*}
          \ket{01} = \ket{\beta_{01}} + \ket{\beta_{11}} \\
          = \frac{1}{\sqrt{2}} (\ket{01} + \ket{10}) + \frac{1}{\sqrt{2}} (\ket{01} - \ket{10})\\
          = \frac{2}{\sqrt{2}} \ket{01} + 0 \ket{10}
        \end{align*}

        \begin{align*}
          \ket{10} = \ket{\beta_{01}} - \ket{\beta_{11}} \\
          = \frac{1}{\sqrt{2}} (\ket{01} + \ket{10}) - \frac{1}{\sqrt{2}} (\ket{01} - \ket{10})\\
          = 0 \ket{01} + \frac{2}{\sqrt{2}} \ket{10} 
        \end{align*}

  %Queston 3
  \item Explain what does it mean to perform a Bell basis measurement of a two-qubit state\dots


  % Question 4
  \item Describe a quantum circuit that starting with \ket{000} prepares the output state:

  \begin{equation*}
    \ket{\psi} = \frac{1}{\sqrt{2}}(\ket{000} + \ket{111}) 
  \end{equation*}

  The following 3 stage circuit will produce the desiried output state:
  \begin{center}
    \begin{quantikz}
      & \gate{H} \slice{1} & \ctrl{1} \slice{2} & & & \\
      & & \targ{} & \ctrl{1} \slice{3} & & \\
      & & & \targ{} & &
    \end{quantikz}
  \end{center}

  \begin{itemize}
    \item Stage 1 - Hadamard Gate: 
      \begin{align*}
        \ket{0} \otimes \ket{0} \otimes \ket{0} \xrightarrow{H \otimes I \otimes I} \frac{\ket{0} + \ket{1}}{\sqrt{2}} \otimes \ket{0} \otimes \ket{0} \\
        = \frac{1}{\sqrt{2}}(\ket{000} + \ket{100}) 
      \end{align*}
    \item Stage 2 - 1$^{\text{st}}$ Control Not:

      Just looking at the first two qubits of the superposition. The first qubit is the control bit and second qubit is the bit we may modifiy.
      \begin{align*}
        & \frac{1}{\sqrt{2}}(\ket{000} + \ket{100}) \xrightarrow{\text{CNOT}} \frac{1}{\sqrt{2}}(\ket{000} + \ket{110}) 
      \end{align*}
    \item Stage 3 - 2$^{\text{nd}}$ Control Not:
      
      Just looking at the last two qubits of the superposition.  The second qubit is the control bit and the third qubit is the bit we may modify
      \begin{align*}
        & \frac{1}{\sqrt{2}}(\ket{000} + \ket{110}) \xrightarrow{\text{CNOT}} \frac{1}{\sqrt{2}}(\ket{000} + \ket{111}) 
      \end{align*}
  \end{itemize}


\end{enumerate}
\end{document}
