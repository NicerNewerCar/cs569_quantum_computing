\documentclass[]{article}
	
\usepackage[margin=1in]{geometry}		% For setting margins
\usepackage{amsmath}
\usepackage{amssymb}
\usepackage{fancyhdr}
\usepackage{graphicx}
\usepackage{cancel}
\usepackage{tikz}
\usetikzlibrary{quantikz2}
\usepackage{braket}

%%%%%%%%%%%%%%%%%%%%%%
% Set up fancy header/footer
\pagestyle{fancy}
\fancyhead[LO,L]{Anthony J. Lombardi}
\fancyhead[CO,C]{CS569: Quantum Info and Computing}
\fancyhead[RO,R]{\today}
\fancyfoot[LO,L]{}
\fancyfoot[CO,C]{\thepage}
\fancyfoot[RO,R]{}
\renewcommand{\headrulewidth}{0.4pt}
\renewcommand{\footrulewidth}{0.4pt}
%%%%%%%%%%%%%%%%%%%%%%

\begin{document}
  \section*{Assignment 4}
    \begin{enumerate}
      \item Spectral Decomposition

      \begin{enumerate}
        \item Spectral decompositions of the obserables X, Z, Y 

          Calculate the eigenvalues of X:
          \begin{gather*}
            det(\lambda I - X)=
            det(
            \begin{bmatrix}
              \lambda & 0 \\
              0 & \lambda 
            \end{bmatrix}
            - \begin{bmatrix}
              0 & 1 \\
              1 & 0
            \end{bmatrix}
            ) = det (
            \begin{bmatrix}
              \lambda & -1\\
              -1 & \lambda
            \end{bmatrix} 
            )\\
            \lambda^2 -1 = 0\\
            \lambda_1 = 1, \lambda_2 = -1
          \end{gather*}

          Calculate the eigenvectors of X:

          \noindent\begin{minipage}{.49\linewidth}
            \begin{gather*}
              (X - \lambda_1 I) \cdot \vec{\lambda_1} = 0\\
              \begin{bmatrix}
                -1 & 1\\
                1 & -1
              \end{bmatrix}
              \cdot 
              \begin{bmatrix}
                x\\ y
              \end{bmatrix}
              = \begin{bmatrix}
                0\\0
              \end{bmatrix}\\
              -x+y=0\\
              x-y=0\\
              x = 1, y = 1\\
              \ket{\lambda_1} = \frac{1}{\sqrt{2}}\begin{bmatrix}
                1\\1
              \end{bmatrix}
            \end{gather*}
          \end{minipage}
          \begin{minipage}{.49\linewidth}
            \begin{gather*}
              (X - \lambda_2 I) \cdot \vec{\lambda_2} = 0\\
              \begin{bmatrix}
                1 & 1\\
                1 & 1
              \end{bmatrix}
              \cdot 
              \begin{bmatrix}
                x\\ y
              \end{bmatrix}
              = \begin{bmatrix}
                0\\0
              \end{bmatrix}\\
              x+y=0\\
              x+y=0\\
              x = 1, y = -1\\
              \ket{\lambda_2} = \frac{1}{\sqrt{2}}\begin{bmatrix}
                1\\-1
              \end{bmatrix}
            \end{gather*}
          \end{minipage}

          Spectral decomposition of X
          \begin{gather*}
            X = \sum_{i=1}^{2} \lambda_i \ket{\lambda_i}\bra{\lambda_i}
            = \lambda_1 \ket{\lambda_1}\bra{\lambda_1} + \lambda_2 \ket{\lambda_2}\bra{\lambda_2}\\
            = (+1) \frac{1}{\sqrt{2}} \begin{bmatrix}1\\1\end{bmatrix}
            \frac{1}{\sqrt{2}} \begin{bmatrix}1&1\end{bmatrix}
            + (-1) \frac{1}{\sqrt{2}} \begin{bmatrix}1\\-1\end{bmatrix}
            \frac{1}{\sqrt{2}} \begin{bmatrix}1&-1\end{bmatrix}\\
            = (+1) \frac{1}{2} \begin{bmatrix}1&1\\1&1\end{bmatrix} 
            + (-1) \frac{1}{2} \begin{bmatrix}1&-1\\-1&1\end{bmatrix}
          \end{gather*}

          Calculate the eigen values of Z:
          \begin{gather*}
            det(\lambda I - Z) = det(
            \begin{bmatrix}
              \lambda & 0\\
              0 & \lambda
            \end{bmatrix}
            - 
            \begin{bmatrix}
              1 & 0\\
              0 & -1
            \end{bmatrix}
            ) = 
            det (
            \begin{bmatrix}
              \lambda - 1 & 0\\
              0 & \lambda + 1
            \end{bmatrix}
            )\\
            (\lambda - 1) (\lambda + 1) = 0\\
            \lambda_1 = 1, \lambda_2 = -1
          \end{gather*}


          \noindent\begin{minipage}{.49\linewidth}
            \begin{gather*}
              (Z - \lambda_1 I) \cdot \vec{\lambda_1} = 0\\
              \begin{bmatrix}
                0 & 0\\
                0 & -2
              \end{bmatrix}
              \cdot 
              \begin{bmatrix}
                x\\ y
              \end{bmatrix}
              = \begin{bmatrix}
                0\\0
              \end{bmatrix}\\
              \text{Apply Gaussian elimination}\\
              \begin{bmatrix}
                0 & 1\\
                0 & 0
              \end{bmatrix}
              \cdot 
              \begin{bmatrix}
                y\\ x
              \end{bmatrix}
              = \begin{bmatrix}
                0\\0
              \end{bmatrix}\\
              y = 0, x = 1\\
              \ket{\lambda_1} = \begin{bmatrix}
                1\\0
              \end{bmatrix}
            \end{gather*}
          \end{minipage}
          \begin{minipage}{.49\linewidth}
            \begin{gather*}
              (Z - \lambda_2 I) \cdot \vec{\lambda_2} = 0\\
              \begin{bmatrix}
                2 & 0\\
                0 & 0
              \end{bmatrix}
              \cdot 
              \begin{bmatrix}
                x\\ y
              \end{bmatrix}
              = \begin{bmatrix}
                0\\0
              \end{bmatrix}\\
              \text{Apply Gaussian elimination}\\
              \begin{bmatrix}
                1 & 0\\
                0 & 0
              \end{bmatrix}
              \cdot 
              \begin{bmatrix}
                x\\ y
              \end{bmatrix}
              = \begin{bmatrix}
                0\\0
              \end{bmatrix}\\
              x = 0, y = 1\\
              \ket{\lambda_2} = \begin{bmatrix}
                0\\1
              \end{bmatrix}
            \end{gather*}
          \end{minipage}

          Spectral decomposition of Z
          \begin{gather*}
            Z = \sum_{i=1}^{2} \lambda_i \ket{\lambda_i}\bra{\lambda_i}
            = \lambda_1 \ket{\lambda_1}\bra{\lambda_1} + \lambda_2 \ket{\lambda_2}\bra{\lambda_2}\\
            = (+1) \begin{bmatrix}1\\0\end{bmatrix} \begin{bmatrix}1&0\end{bmatrix}
            + (-1) \begin{bmatrix}0\\1\end{bmatrix} \begin{bmatrix}0&1\end{bmatrix}\\
            = (+1) \begin{bmatrix}1&0\\0&0\end{bmatrix} + (-1) \begin{bmatrix}0&0\\0&1\end{bmatrix}
          \end{gather*}

          Calculate the eigenvalues of Y
          \begin{gather*}
            det(\lambda I - Y) = det(
            \begin{bmatrix}
              \lambda & 0\\
              0 & \lambda
            \end{bmatrix}
            - 
            \begin{bmatrix}
              0 & -i\\
              i & 0
            \end{bmatrix}
            ) = 
            det (
            \begin{bmatrix}
              \lambda & i\\
              -i & \lambda 
            \end{bmatrix}
            )\\
            \lambda^2 - (-i*i) = 0\\
            \lambda^2 - 1 = 0\\
            \lambda_1 = 1, \lambda_2 = -1
          \end{gather*}

          \noindent\begin{minipage}{.49\linewidth}
            \begin{gather*}
              (Y - \lambda_1 I) \cdot \vec{\lambda_1} = 0\\
              \begin{bmatrix}
                -1 & -i\\
                i & -1
              \end{bmatrix}
              \cdot 
              \begin{bmatrix}
                x\\ y
              \end{bmatrix}
              = \begin{bmatrix}
                0\\0
              \end{bmatrix}\\
              \text{Apply Gaussian elimination}\\
              \begin{bmatrix}
                1 & i\\
                0 & 0
              \end{bmatrix}
              \cdot 
              \begin{bmatrix}
                x\\ y
              \end{bmatrix}
              = \begin{bmatrix}
                0\\0
              \end{bmatrix}\\
              x = -i, y = 1\\
              \ket{\lambda_1} = \frac{1}{\sqrt{2}} \begin{bmatrix}
                -i\\1
              \end{bmatrix}
            \end{gather*}
          \end{minipage}
          \begin{minipage}{.45\linewidth}
            \begin{gather*}
              (Y - \lambda_2 I) \cdot \vec{\lambda_2} = 0\\
              \begin{bmatrix}
                1 & -i\\
                i & 1
              \end{bmatrix}
              \cdot 
              \begin{bmatrix}
                x\\ y
              \end{bmatrix}
              = \begin{bmatrix}
                0\\0
              \end{bmatrix}\\
              \text{Apply Gaussian elimination}\\
              \begin{bmatrix}
                1 & -i\\
                0 & 0
              \end{bmatrix}
              \cdot 
              \begin{bmatrix}
                x\\ y
              \end{bmatrix}
              = \begin{bmatrix}
                0\\0
              \end{bmatrix}\\
              x = i, y = 1\\
              \ket{\lambda_2} = \frac{1}{\sqrt{2}} \begin{bmatrix}
                i\\1
              \end{bmatrix}
            \end{gather*}
          \end{minipage}

          Spectral decomposition of Y
          \begin{gather*}
            Y = \sum_{i=1}^{2} \lambda_i \ket{\lambda_i}\bra{\lambda_i}
            = \lambda_1 \ket{\lambda_1}\bra{\lambda_1} + \lambda_2 \ket{\lambda_2}\bra{\lambda_2}\\
            = (+1) \frac{1}{\sqrt{2}} \begin{bmatrix}-i\\1\end{bmatrix}
            \frac{1}{\sqrt{2}} \begin{bmatrix}-i&1\end{bmatrix}
            + (-1) \frac{1}{\sqrt{2}} \begin{bmatrix}i\\1\end{bmatrix}
            \frac{1}{\sqrt{2}} \begin{bmatrix}i&1\end{bmatrix}\\
            = (+1) \frac{1}{2} \begin{bmatrix}-1&-i\\-i&1\end{bmatrix} 
            + (-1) \frac{1}{2}\begin{bmatrix}-1&i\\i&1\end{bmatrix}
          \end{gather*}

        \item Show that $\braket{\psi|P_r|\psi} \ge 0$ and moreover, $\sum_r \braket{\psi|P_r|\psi} = 1$

          We know that $P^2 = P$ for any projector P. We know that $P^{\intercal}$ = P.
          We also know that the inner product of any normalized state is 1. Since 
          
          \begin{gather*}
            \ket{\psi} = \alpha\ket{0} + \beta\ket{1}\\
            \braket{\psi|\psi} = \begin{pmatrix} \alpha \\ \beta \end{pmatrix}
            \begin{pmatrix} \alpha & \beta \end{pmatrix}
            = |\alpha|^2 + |\beta|^2 = 1
          \end{gather*}

          So we can say that
          \begin{gather*}
            \braket{\psi | P_r | \psi} = \braket{\psi | P_r^{\intercal} P_r | \psi}
            = (\bra{\psi}P_r^{\intercal})(P_r \ket{\psi}) = \braket{\phi | \phi} \ge 0\\
            \therefore \sum_r \braket{\psi | P_r | \psi} = \braket{\psi | \sum_r P_r | \psi}
            = \braket{\psi | I | \psi} = \braket{\psi | \psi}= 1
          \end{gather*}

          

        \item Compute $\braket{X}$ and  $\braket{Z}$ under $\ket{+}$

        Compute $\braket{X}$ under $\ket{+}$

        \begin{gather*}
          \braket{X}_{\ket{+}} = \braket{+|X|+} = \braket{+|(\ket{+}\bra{+} - \ket{-}\bra{-})|+}\\
          = \braket{+|+}\braket{+|+} - \braket{+|-}\braket{-|+}\\
          = 1 * 1 - 0 * 0 = 1
        \end{gather*}

        Compute $\braket{Z}$ under $\ket{+}$

        \begin{gather*}
          \braket{Z}_{\ket{+}} = \braket{+|Z|+} = \braket{+|(\ket{0}\bra{0} - \ket{1}\bra{1})|+}\\
          = \braket{+|0}\braket{0|+} - \braket{+|1}\braket{1|+}\\
          = (\frac{1}{\sqrt{2}} * \frac{1}{\sqrt{2}}) - (\frac{1}{\sqrt{2}} * \frac{1}{\sqrt{2}})
          = \frac{1}{2} - \frac{1}{2} = 0
        \end{gather*}


        \item Compute $\braket{X_1Z_2}$ under $\ket{\beta_{00}}$
          \begin{gather*}
            \braket{X_1 Z_2}_{\ket{\beta_{00}}} = \frac{1}{2}[\bra{00} + \bra{11}(X_1 Z_2) \ket{00} + \ket{11}]\\
            = \frac{1}{2}[\braket{00|X_1 Z_2|00} + \braket{00|X_1 Z_2|11}
            + \braket{11|X_1 Z_2|00} + \braket{1|X_1 Z_2|11}]\\
            = \frac{1}{2}[0*1 + 1*0 + 1*0 + 0*-1] = 0
          \end{gather*}
      \end{enumerate}
    \item CHSH Game

      \begin{enumerate}
        \item (T/F)$\braket{\mathcal O}$ holds. True.
        \item (T/F) $\braket{QS}$ = $\frac{1}{\sqrt{2}}$.
          \begin{gather*}
            QS = Z_1 \otimes \frac{-1}{\sqrt{2}}(Z_2 + X_2) = \frac{-1}{\sqrt{2}}(Z_1 Z_2 + Z_1 X_2)\\
            \braket{QS}_{\ket{\beta_{11}}} = \braket{\beta_{11} | QS | \beta_{11}}
            = \frac{\bra{01} - \bra{10}}{\sqrt{2}}[\frac{-1}{\sqrt{2}}(Z_1 Z_2 + Z_1 X_2)]
            \frac{\ket{01} - \ket{10}}{\sqrt{2}}\\
            = \frac{-1}{2\sqrt{2}}[(\bra{01} - \bra{10})Z_1 Z_2(\ket{01} - \bra{10}) + 
            (\bra{01} - \bra{10})Z_1 X_2(\ket{01} - \bra{10})]\\
            = \frac{-1}{2\sqrt{2}}[
            \braket{01|Z_1 Z_2|01} - \braket{01|Z_1 Z_2|10} - \braket{10|Z_1 Z_2|01} + \braket{10|Z_1 Z_2|10}\\
            + \braket{01|Z_1 X_2|01} - \braket{01|Z_1 X_2|10} - \braket{10|Z_1 X_2|01} + \braket{10|Z_1 X_2|10}]\\
            = \frac{-1}{2\sqrt{2}}[-1 -0 -0 + (-1) + 0-0-0+0] = \frac{-1}{2\sqrt{2}}*-2 = \frac{1}{\sqrt{2}}
          \end{gather*}
          
          True.
        \item (T/F) $\braket{RS}$ = $\frac{1}{\sqrt{2}}$.
          \begin{gather*}
            RS = X_1 \otimes \frac{-1}{\sqrt{2}}(Z_2 + X_2) = \frac{-1}{\sqrt{2}}(X_1 Z_2 + X_1 X_2)\\
            \braket{RS}_{\ket{\beta_{11}}} = \braket{\beta_{11} | RS | \beta_{11}}
            = \frac{\bra{01} - \bra{10}}{\sqrt{2}}[\frac{-1}{\sqrt{2}}(X_1 Z_2 + X_1 X_2)]
            \frac{\ket{01} - \ket{10}}{\sqrt{2}}\\
            = \frac{-1}{2\sqrt{2}}[(\bra{01} - \bra{10})X_1 Z_2(\ket{01} - \bra{10}) + 
            (\bra{01} - \bra{10})X_1 X_2(\ket{01} - \bra{10})]\\
            = \frac{-1}{2\sqrt{2}}[
            \braket{01|X_1 Z_2|01} - \braket{01|X_1 Z_2|10} - \braket{10|X_1 Z_2|01} + \braket{10|X_1 Z_2|10}\\
            + \braket{01|X_1 X_2|01} - \braket{01|X_1 X_2|10} - \braket{10|X_1 X_2|01} + \braket{10|X_1 X_2|10}]\\
            = \frac{-1}{2\sqrt{2}}[0-0-0+0 + 0 - 1 -1 +0] = \frac{-1}{2\sqrt{2}}*-2 = \frac{1}{\sqrt{2}}
          \end{gather*}

          True.
        \item (T/F) $\braket{RT}$ = $\frac{1}{\sqrt{2}}$.
          \begin{gather*}
            RT = X_1 \otimes \frac{1}{\sqrt{2}}(Z_2 - X_2) = \frac{1}{\sqrt{2}}(X_1 Z_2 - X_1 X_2)\\
            \braket{RT}_{\ket{\beta_{11}}}
            = \frac{1}{2\sqrt{2}}[(\bra{01} - \bra{10})X_1 Z_2(\ket{01}-\ket{10})
            - (\bra{01} - \bra{10})X_1 X_2(\ket{01}-\ket{10})]\\
            \text{From parts b and d we know that} \braket{X_1Z_2}_{\ket{\beta_{11}}} = 0 
            \text{ and that} \braket{X_1X_2}_{\ket{\beta_{11}}} = -2 \text{.}\\
            = \frac{1}{2\sqrt{2}}[0 - (-2)] = \frac{1}{2\sqrt{2}} * 2 = \frac{1}{\sqrt{2}} 
          \end{gather*}

          True.
        \item (T/F) $\braket{QT}$ = $-\frac{1}{\sqrt{2}}$.
          \begin{gather*}
            QT = Z_1 \otimes \frac{1}{\sqrt{2}}(Z_2 - X_2) = \frac{1}{\sqrt{2}}(Z_1 Z_2 - Z_1 X_2)\\
            \braket{QT}_{\ket{\beta_{11}}}
            = \frac{1}{2\sqrt{2}}[(\bra{01} - \bra{10})Z_1 Z_2(\ket{01}-\ket{10})
            - (\bra{01} - \bra{10})Z_1 X_2(\ket{01}-\ket{10})]\\
            \text{From parts b and d we know that} \braket{Z_1Z_2}_{\ket{\beta_{11}}} = -2 
            \text{ and that} \braket{Z_1X_2}_{\ket{\beta_{11}}} = 0 \text{.}\\
            = \frac{1}{2\sqrt{2}}[(-2) - 0] = \frac{1}{2\sqrt{2}} * -2 = \frac{-1}{\sqrt{2}} 
          \end{gather*}

          True.
        \item (T/F) If Q, R, S, T are random variables with $\pm$ 1 values then 

          \noindent$\mathbb{E}$[QS] + $\mathbb{E}$[RS] + $\mathbb{E}$[RT] - $\mathbb{E}$[QT] $\le$ 2.

          \begin{gather*}
            QS + RS + RT - QT\\
            (Q+R)S + (R-Q)T
          \end{gather*}
            Either Q+R or R-Q evaluates to zero, then the other term must evaluate to $\pm$ 2.
            Then the whole expression must evaluate to $\pm$ 2. $\therefore$ True, the expression evaluates to $\le$ 2.

        \item (T/F) $\braket{\mathcal O} = 2\sqrt{2}$ (Does this mean Quantum 1, Einstein 0?)
          \begin{gather*}
            \braket{\mathcal O} = \braket{QS} + \braket{RS} + \braket{RT} - \braket{QT}\\
            = \frac{1}{\sqrt{2}} + \frac{1}{\sqrt{2}} + \frac{1}{\sqrt{2}} - \frac{-1}{\sqrt{2}} = \frac{4}{\sqrt{2}}
            = 2\sqrt{2}
          \end{gather*}

          True, this does mean Quantum 1, Einstein 0, since we were able to calculate a value that is $>$ 2 
          using the quantum states.
      \end{enumerate}
    \end{enumerate}
\end{document} 

